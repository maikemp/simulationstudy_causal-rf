\documentclass[11pt, a4paper, leqno]{article}
\usepackage{a4wide}
\usepackage[T1]{fontenc}
\usepackage[utf8]{inputenc}
\usepackage{float, afterpage, rotating, graphicx}
\usepackage{epstopdf}
\usepackage{longtable, booktabs, tabularx}
\usepackage{fancyvrb, moreverb, relsize}
\usepackage{eurosym, calc}
% \usepackage{chngcntr}
\usepackage{amsmath, amssymb, amsfonts, amsthm, bm, MnSymbol}
\usepackage{caption}
\usepackage{mdwlist}
\usepackage{xfrac}
\usepackage{setspace}
\usepackage{xcolor}
\usepackage{subcaption}
\usepackage{minibox}
% \usepackage{pdf14} % Enable for Manuscriptcentral -- can't handle pdf 1.5,
% \usepackage{endfloat} % Enable to move tables / figures to the end. Useful for some submissions.



\usepackage{natbib}
\bibliographystyle{rusnat}




\usepackage[unicode=true]{hyperref}
\hypersetup{
    colorlinks=true,
    linkcolor=black,
    anchorcolor=black,
    citecolor=black,,
    filecolor=black,
    menucolor=black,
    runcolor=black,
    urlcolor=black
}


\widowpenalty=10000
\clubpenalty=10000

\setlength{\parskip}{1ex}
\setlength{\parindent}{0ex}
\setstretch{1.5}


\begin{document}

\title{Simulation Study on the Coverage Probabilities of Confidence Intervals for Heterogeneous Treatment Effects Estimated with Causal Random Forests\thanks{Maike Metz-Peeters, Universität Bonn. Email: \href{mailto:maike-m-p@web.de}{\nolinkurl{maike-m-p [at] web [dot] de}}.}}

\author{Maike Metz-Peeters}

\date{
{\bf Preliminary -- please do not quote} 
\\[1ex] 
\today
}

\maketitle


\begin{abstract}
	In this simulation study, I investigate the coverage probabilities of the confidence intervals for treatment effects estimated by a nonparametric causal forest developed by \citet{wa18}. I replicate there results and ... 
    (Shortly state results.)
    
\end{abstract}
\clearpage

\section{Introduction} % (fold)
\label{sec:introduction}

will now first give a brief insight into the method itself, before describing my approach for the simulation study, aftre which I will present my results. It follows a short conclusion.


If you are using this template, please cite this item from the references: \citet{GaudeckerEconProjectTemplates}


\section{Background} % (fold)
\label{sec:background}
The method described in \citet{wa18} is an attempt to estimate heterogeneous treatment effects in a highly data-driven way while still enabling valid statistical inference. This makes it possible to discover even unexpected structures in the data. In the following I will give a short description of this method following \citet{wa18} unless stated otherwise.

For this purpose, the authors assume having at hand \(n\) i.i.d. draws of a feature vector \(X_i \in [0,1]^d\) or following any other density bounded away from 0 and infinity, a treatment indicator \(W_i \in \{0,1\}\), and an outcome \(Y_i \in \mathbb{R}\).
The treatment effect is the expectation of the difference between the outcome under treatment and without treatment:

\[\tau(x)=\mathbb{E}[Y_i^{(1)}-Y_i^{(0)}|X_i=x]\]

where \(Y_i^{(1)}\) denotes the outcome if treatment has taken place and \(Y_i^{(0)}\) is the outcome without treatment.
While heterogeneity in the treatment effect depending on the observable variables is allowed, unconfoundedness is assumed:

\[\{Y_i^{0},Y_i^{1}\} \upmodels W_i|X_i. \]

This means that the selection bias into treatment disappears when controlling for observable features, and thus around observations that are "close" in the feature space behave like a randomized experiment. Nearest neighbor methods exploit this assumption
\(\) \(\) \(\) 


\[\hat{\mu}(x)= \frac{1}{|\{i:X_i \in L(x)\}|}\sum_{i:X_i \in L(x)}Y_i\]





 \(\) \(\) \(\) \(\) \(\) 


\section{Simulation Study} % (fold)
\label{sec:simulation}

\subsection{Approach} % (fold)
\label{sec:sim_approach}



\subsection{Results} % (fold)
\label{sec:sim_results}
    
\input{../../out/tables/coverage_table_setup_1.tex}



\section{Conclusion} % (fold)
\label{sec:conclusion}

\begin{figure}
    \caption{Plot of Micro data for one forest and one testset of Setup 4 with d=3}
    
    \includegraphics[width=\textwidth]{../../out/figures/te_plot_setup_4}

\end{figure}




\pagebreak
% section introduction (end)





\bibliography{refs}



% \appendix

% The chngctr package is needed for the following lines.
% \counterwithin{table}{section}
% \counterwithin{figure}{section}

\end{document}
